% ****** Start of file apssamp.tex ******
%
%   This file is part of the APS files in the REVTeX 4.1 distribution.
%   Version 4.1r of REVTeX, August 2010
%
%   Copyright (c) 2009, 2010 The American Physical Society.
%
%   See the REVTeX 4 README file for restrictions and more information.
%
% TeX'ing this file requires that you have AMS-LaTeX 2.0 installed
% as well as the rest of the prerequisites for REVTeX 4.1
%
% See the REVTeX 4 README file
% It also requires running BibTeX. The commands are as follows:
%
%  1)  latex apssamp.tex
%  2)  bibtex apssamp
%  3)  latex apssamp.tex
%  4)  latex apssamp.tex
%
\documentclass[%
 reprint,
%superscriptaddress,
%groupedaddress,
%unsortedaddress,
%runinaddress,
%frontmatterverbose, 
%preprint,
%showpacs,preprintnumbers,
%nofootinbib,
%nobibnotes,
%bibnotes,
 amsmath,amssymb,
 aps,
%pra,
%prb,
%rmp,
%prstab,
%prstper,
%floatfix,
]{revtex4-1}

\usepackage{graphicx}% Include figure files
\usepackage{dcolumn}% Align table columns on decimal point
\usepackage{bm}% bold math
%\usepackage{hyperref}% add hypertext capabilities
%\usepackage[mathlines]{lineno}% Enable numbering of text and display math
%\linenumbers\relax % Commence numbering lines

%\usepackage[showframe,%Uncomment any one of the following lines to test 
%%scale=0.7, marginratio={1:1, 2:3}, ignoreall,% default settings
%%text={7in,10in},centering,
%%margin=1.5in,
%%total={6.5in,8.75in}, top=1.2in, left=0.9in, includefoot,
%%height=10in,a5paper,hmargin={3cm,0.8in},
%]{geometry}

\begin{document}

% \preprint{APS/123-QED}

\title{Quantum-like behavior at Macroscopic Scale (Analysis)}% Force line breaks with \\
% \thanks{A footnote to the article title}%

\author{Marcel Reis Soubkovsky}
 \email{marcel.soubkovsky4@etu.univ-lorraine.fr}
\affiliation{%
 University of Lorraine, Master's in Applied Physics and Physics Engineering
}%

\date{\today}% It is always \today, today,
             %  but any date may be explicitly specified

\begin{abstract}
% An article usually includes an abstract, a concise summary of the work
% covered at length in the main body of the article. 
Write this at last
\begin{description}
\item[Structure]
% I still don't know what to put here
\end{description}
\end{abstract}

\maketitle

%\tableofcontents

\section{Physical Phenomenon}
A bath filled with a fluid can hold a droplet of the same fluid when put into vertical vibration.\cite{couder_bouncing_2005} The droplet is kept bouncing over the fluid 

    \subsection{Faraday Wave and Threshold}
        Show the functioning of the theory that enables us to understand it.
    \subsection{Experimental Apparatus used by Harris, Moukhtar, Fort, Couder, and Bush}


\section{Similarities to Quantum Phenomena}

    \subsection{Patterns formed by free moving droplets}
    \subsection{Diffraction and Interference of bouncing droplet}
        \subsubsection{Single and Double Slit}
    \subsection{Circular Corral}
    
    
    
    \bibliographystyle{aipauth4-1}
    \bibliography{bibtex}



\end{document}
%